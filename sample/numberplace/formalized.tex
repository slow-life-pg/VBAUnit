% !TeX program = lualatex
\documentclass[a4paper,11pt]{article}

\usepackage{amsmath,amssymb,amsthm,mathtools}
\usepackage{algorithm}
\usepackage[noend]{algpseudocode}
\usepackage{hyperref}
\usepackage{geometry}
\geometry{margin=25mm}

% argmin: subscript applies only to "min"
\newcommand{\argmin}{\operatorname{arg}\,\min}

\title{Mathematical Formalization of the Number Place Solving Algorithm}
\author{}
\date{}

\begin{document}
\maketitle

\begin{abstract}
This paper presents a formal mathematical formulation of a backtracking-based algorithm for solving the Number Place (Sudoku) puzzle.
The algorithm is modeled as a constraint satisfaction problem (CSP), and the process of candidate generation, constraint propagation, and backtracking search is rigorously described.
\end{abstract}

\section{Problem Formulation}

\subsection{Grid and Variables}
Let \(I = \{1, \dots, 9\}\) be the set of row and column indices.
Each cell of the \(9 \times 9\) grid is represented by a variable \(X_{r,c}\), where \(r, c \in I\).
The collection of all variables is
\[
\mathcal{V} = \{ X_{r,c} \mid r, c \in I \}.
\]
Given digits (the clues) are represented by the set
\[
\mathrm{Givens} \subseteq I \times I \times I,
\]
and the initial domain of each variable is defined as
\[
D_{r,c}^{(0)} =
\begin{cases}
\{v\} & \text{if } (r,c,v) \in \mathrm{Givens},\\[4pt]
\{1, \dots, 9\} & \text{otherwise.}
\end{cases}
\]

\subsection{Block Structure}
Define the block index functions \(B(r) = \lceil r/3 \rceil\) and \(B(c) = \lceil c/3 \rceil\).
Each \(3 \times 3\) block is represented as
\[
\mathrm{Blk}(b_r,b_c) = \{ (r,c) \in I^2 \mid B(r)=b_r,\, B(c)=b_c \}.
\]

\subsection{Constraints}
The CSP is subject to the following constraints:
\begin{align*}
&\text{(Row constraint)}   && \forall r, v:\ |\{c \mid X_{r,c}=v\}| \le 1, \\
&\text{(Column constraint)}&& \forall c, v:\ |\{r \mid X_{r,c}=v\}| \le 1, \\
&\text{(Block constraint)} && \forall b_r,b_c,v:\ |\{(r,c)\in \mathrm{Blk}(b_r,b_c) \mid X_{r,c}=v\}| \le 1, \\
&\text{(Domain constraint)}&& \forall r,c:\ X_{r,c} \in \{1,\dots,9\}.
\end{align*}

\section{Constraint Propagation and Candidate Refinement}

\subsection{State Representation}
At any stage \(t\), the algorithm maintains a family of candidate domains
\[
\mathcal{D}^{(t)} = \{ D_{r,c}^{(t)} \subseteq \{1, \dots, 9\} \}_{r,c \in I}.
\]
A \emph{satisfied state} occurs when all \(D_{r,c}^{(t)}\) are singletons and all constraints are satisfied.
If any \(D_{r,c}^{(t)} = \varnothing\), the state is inconsistent.

\subsection{Forward Elimination}
When a cell \((r,c)\) is assigned \(v\), the candidate sets are updated as:
\begin{align*}
D_{r,c'} &\gets D_{r,c'} \setminus \{v\}, && \forall c' \neq c,\\
D_{r',c} &\gets D_{r',c} \setminus \{v\}, && \forall r' \neq r,\\
D_{r',c'} &\gets D_{r',c'} \setminus \{v\}, && \forall (r',c') \in \mathrm{Blk}(B(r),B(c)),\ (r',c')\neq(r,c).
\end{align*}
If any domain becomes a singleton, its unique value is immediately propagated using the same rule (this corresponds to forward checking).

\subsection{Block-Based Candidate Initialization}
For each block \(\mathrm{Blk}(b_r,b_c)\), define the set of unused digits:
\[
U_{b_r,b_c} = \{1,\dots,9\} \setminus \{ X_{r,c} \mid (r,c)\in \mathrm{Blk}(b_r,b_c), |D_{r,c}|=1 \}.
\]
Each undecided cell in the block initially receives candidates from \(U_{b_r,b_c}\), which are then refined by removing digits already present in the same row or column.

\subsection{Unique Occurrence Rule}
For each block (and analogously for rows and columns), define
\[
S_v = \{ (r,c) \in \mathrm{Blk}(b_r,b_c) \mid v \in D_{r,c} \}.
\]
If \(|S_v|=1\), the value \(v\) is fixed at that cell.  
This rule corresponds to identifying \emph{hidden singles}.

\section{Search and Backtracking}

\subsection{Variable Selection}
Let \(U = \{(r,c) \mid |D_{r,c}| > 1\}\) be the set of unsolved cells.
The algorithm selects the variable with the fewest candidates (Minimum Remaining Values heuristic):
\[
(r^{*}, c^{*}) \in \argmin_{(r,c)\in U} |D_{r,c}|.
\]
If multiple cells share the same minimal size, one is chosen at random.

\subsection{Backtracking and Forward Checking}
For each \(v \in D_{r^{*},c^{*}}\), assign \(X_{r^{*},c^{*}}=v\) and apply constraint propagation to obtain an updated domain family \(\mathcal{D}'\).
If any domain becomes empty, the branch fails and the algorithm tries the next candidate.
Otherwise, recursion continues on \(\mathcal{D}'\).

\subsection{Termination}
If all domains become singletons, a valid solution is found.
If all candidates have been tested and failed, the algorithm backtracks to the previous step.
To detect multiple solutions, the search would continue beyond the first solution, which is omitted in the present version.

\section{Formal Algorithm}

\begin{algorithm}[h!]
\caption{Recursive Sudoku Solver}
\begin{algorithmic}[1]
\Function{Solve}{$\mathcal{D}$}
    \State \Call{Propagate}{$\mathcal{D}$}  \Comment{Apply rules in Sections 2.2–2.4}
    \If{$\exists (r,c): D_{r,c} = \varnothing$} \State \Return FAIL \EndIf
    \If{$\forall (r,c): |D_{r,c}|=1$} \State \Return $\mathcal{D}$ \Comment{Solution found} \EndIf
    \State Select $(r^{*}, c^{*}) = \argmin_{(r,c): |D_{r,c}|>1} |D_{r,c}|$
    \ForAll{$v \in D_{r^{*},c^{*}}$}
        \State $\mathcal{D}' \gets \text{Clone}(\mathcal{D})$
        \State $D'_{r^{*},c^{*}} \gets \{v\}$
        \State $R \gets$ \Call{Solve}{$\mathcal{D}'$}
        \If{$R \neq \text{FAIL}$} \State \Return $R$ \EndIf
    \EndFor
    \State \Return FAIL
\EndFunction
\end{algorithmic}
\end{algorithm}

\section{Correctness and Complexity}

\subsection{Soundness}
Each propagation step (row, column, and block elimination) removes only values inconsistent with the Sudoku constraints.
Thus, any solution found by the algorithm satisfies all constraints.

\subsection{Completeness}
Because every unassigned cell is eventually tested with all possible candidates through backtracking, any existing solution will eventually be discovered.

\subsection{Computational Complexity}
In the worst case, the search tree has branching factor at most \(9\) and depth up to \(81\), yielding \(O(9^{81})\) complexity.
However, constraint propagation and MRV selection drastically reduce the effective search space in practice.

\section{Conclusion}
We have presented a formal description of the Number Place solving process as a constraint satisfaction problem with heuristic-guided backtracking.
This formulation enables both theoretical analysis and efficient implementation.

\end{document}
